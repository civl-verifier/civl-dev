\documentclass[runningheads,orivec]{llncs}

\usepackage{xcolor}
\usepackage{hyperref}
\hypersetup{
  colorlinks,
  linkcolor={blue!80!black},
  citecolor={blue!80!black},
  urlcolor={blue!80!black}
}

% \renewcommand\UrlFont{\color{blue}\rmfamily}

\usepackage{graphicx}

%%%%% Math %%%%%
\usepackage{amsmath}
\usepackage{amssymb}
\usepackage{stmaryrd}
\SetSymbolFont{stmry}{bold}{U}{stmry}{m}{n} % get rid of bold font shape warning
\usepackage{empheq}

% replace the ugly \emptyset symbol with \varnothing from amssymb
\let\emptyset\varnothing

\newcommand{\choice}{\bullet}
\newcommand{\set}[1]{\{ #1 \}}
\newcommand{\sem}[1]{\llbracket #1 \rrbracket}
\newcommand{\asem}[1]{\llparenthesis #1 \rrparenthesis}

\newcommand{\assume}{\mathrm{assume}}
\newcommand{\assert}{\mathrm{assert}}

\newcommand{\true}{\mathit{true}}
\newcommand{\false}{\mathit{false}}

\newcommand{\fail}{\lightning}


\begin{document}

\title{Efficient Verification-Condition Generation for Atomic Actions}

% \orcidID{0000-1111-2222-3333}
\author{%
Soroush Ebadian\inst{1} \and
Thomas A. Henzinger\inst{1} \and
Bernhard Kragl\inst{1} \and
Shaz Qadeer\inst{2}}

\authorrunning{S. Ebadian \and T.A. Henzinger \and B. Kragl \and S. Qadeer}

\institute{%
IST Austria \and
Facebook}

\maketitle

\section{Action Language}

\paragraph{Syntax.}
%
An \emph{action} is a tuple $(G,I,H,P,O,c)$, where $G$, $I$, $H$, and $P$ are disjoint finite sets of \emph{global}, \emph{input}, \emph{history}, and \emph{prophecy variables}, respectively, $O \subseteq H$ is a set of \emph{output variables}, and $c$ is a command of the form
%
\begin{align*}
  c ::= h := E \mid p =: E \mid \assume\ E \mid \assert\ E \mid c ; c \mid c \choice c,
\end{align*}
%
where $h$ is a history variable, $p$ is a prophecy variable, and $E$ is an expression of appropriate type.
Let $V = G \cup I \cup H \cup P$ be the set of variables of an action.

\paragraph{Semantics.}
%
Let $D$ be a set of values (that includes the Booleans).
A \emph{state} is either a mapping from $V$ to $D$, or the unique \emph{failure state} $\fail$.
For a state $s$ and an expression $E$ we denote $s(E)$ the value of $E$ in $s$.
The semantics $\sem{c}$ of a command $c$ is defined as a binary relation over states as follows:\footnote{$\circ$ denotes relation composition.}
%
\begin{align*}
  \sem{h := E} &= \set{(s,s') \mid s' = s[h \mapsto s(E)]}; \\
  \sem{p =: E} &= \set{(s,s') \mid s = s'[p \mapsto s'(E)]}; \\
  \sem{\assume\ E} &= \set{(s,s) \mid s(E) = \true]}; \\
  \sem{\assert\ E} &= \set{(s,s) \mid s(E) = \true]} \cup \set{(s,\fail) \mid s(E) = \false}; \\
  \sem{c_1 ; c_2} &= \sem{c_1} \circ (\sem{c_2} \cup \set{(\fail,\fail)}); \\
  \sem{c_1 \choice c_2} &= \sem{c_1} \cup \sem{c_2}. \\
\end{align*}

We write $\asem{c}$ for $\sem{c}$ with the pre-state restricted to $G \cup I$ and the post-state restricted to $G \cup O$, i.e., $\asem{c} = \set{(s|_{G \cup I},s'|_{G \cup O}) \mid (s,s') \in \sem{c}}$.

Two actions $(G_1,I_1,H_1,P_1,O_1,c_1)$ and $(G_2,I_2,H_2,P_2,O_2,c_2)$ are \emph{equivalent} if $G_1=G_2$, $I_1=I_2$, $O_1=O_2$, and $\asem{c_1}=\asem{c_2}$.

\paragraph{Transition formula.}
%
We are interested in computing logical formulas $\rho(G,I)$ and $\tau(G,I,G',O')$ such that
\begin{align}
  s \models \rho    &\iff (s,\fail) \not\in \asem{c}; \\
  s,s' \models \tau &\iff (s,s') \in \asem{c}. \label{eq:tau}
\end{align}
%
In \eqref{eq:tau} we mean that the values for $G \cup I$ are taken from $s$ and the values for $G' \cup O'$ are taken from $s'$.

\begin{example}
The following two programs are equivalent and have the transition formula on the right.
\begin{center}
\begin{minipage}{3cm}
\begin{tabbing}
\textbf{hvar} l \\
assume $l > 0$ \\
$x := l$
\end{tabbing}
\end{minipage}
\hspace{1cm}
\begin{minipage}{3cm}
\begin{tabbing}
\textbf{pvar} p \\
$x := p$ \\
assume $x > 0$ \\
$p =: x$
\end{tabbing}
\end{minipage}
\hspace{1cm}
\begin{minipage}{3cm}
\begin{tabbing}
$\rho = true$ \\
$\tau = x' > 0$
\end{tabbing}
\end{minipage}
\end{center}
\end{example}

\bibliographystyle{splncs04}
\bibliography{dblp}

\end{document}
